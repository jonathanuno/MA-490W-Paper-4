\documentclass[letterpaper,oneside,12pt]{report}

\usepackage[USenglish]{babel} %francais, polish, spanish, ...
\usepackage[T1]{fontenc}
\usepackage[margin=1in]{geometry}
\usepackage[ansinew]{inputenc}
\usepackage{setspace}
\usepackage[hidelinks]{hyperref}
\usepackage{amsfonts}
\usepackage{booktabs}
\usepackage{float}
\usepackage{titlesec}
\titlespacing*{\chapter}{0pt}{-10pt}{15pt}

\usepackage{textcomp}
\usepackage{titlesec}
\titleformat{\chapter}{\normalfont\huge}{\thechapter.}{15pt}{\huge}

\doublespacing 

\begin{document}

\pagestyle{empty} 

\title{Lunch Crunch}
\author{Jonathan Chu}
\maketitle

\tableofcontents 
\cleardoublepage 

\pagestyle{plain} 
%\vspace{-5cm}
\chapter{Summary}\label{Summary}
School lunches were established nationwide at the federal level in the United States through the National School Lunch Act in 1945. The original goal of the Act was "to safeguard the health and well-being of the Nation's children\ldots (by) expansion of nonprofit school lunch programs.\cite{NSLA}" Today, this goal goes beyond making sure every child receives something to eat at lunch. Legislation such as the Healthy, Hunger-Free Kids Act of 2010 aims to "make real reforms to the school lunch and breakfast programs by improving the critical nutrition and hunger safety net for millions of children \cite{USDAHHKA}."

The passage of such legislation along with new scientific research on childhood nutrition has forced policymakers to balance three seemingly opposed forces: Students who are concerned with taste and quantity, Schools who are concerned with cost, and the Federal Government who is concerned with developing lifelong healthy eating habits. 

This report addresses the guidelines for the creation of a school lunch that compromises on all three fronts. In order to do so, certain guiding principles and assumptions were made. It is important to understand that a per student customized school lunch is very difficult to create and is outside the scope of this report. A school serving lunches designed under the guidance of this report will ideally serve multiple options (for variety) that all meet a very similar nutritional profile. This universal meal profile will satisfy the majority of students requirements while being cost effective for the school. 

The main tools used in this report were studies published by the United States government. Data on high school students aged 14-18 was used to model a single nutritional profile that would satisfy the most students.  

The result of this report is very flexible and affords schools some independence in operation. Any food that meets the criteria is eligible to be used as part of a school lunch

\chapter{Introduction}\label{Introduction}
\section{Problem Restatement}\label{Problem Restatement}
This report seeks to answer the following questions: 
\begin{itemize}
	\item Develop a mathematical model that takes as input many students individual attributes and outputs a set of standard lunches that will be satisfying\footnote{Satisfied in this report will mean that the meal will have sufficient taste, quantity, nutritional value, and affordability.} to most students. 
	\item  If every student eats a standard school lunch, what percentage of students will receive a satisfactory lunch?
	\item Develop a satisfactory lunch plan (using food categories) that stays within the \$6 budget.
\end{itemize}

\section{Global Assumptions}\label{Global Assumptions}
Creating a plan that answers the above questions is essentially a smaller version of a plan to solve the lingering problem of world hunger. It is necessary to restrict the scope of this report to maintain relevance and to help ensure feasibility.

It is impossible to create an individualized lunch for each student. The logistics required to survey each student and serve them a customized lunch would cost more than many schools could afford.  Models that take into account different attributes such as BMI, level of activity, or lifestyle to output a particular meal become useless if these attributes can not be ascertained or if there is no means of serving such customized meals. The cost of buying specific foods, hiring extra staff, and buying equipment also rises with each individualized meal. Thus this report will address the three main concerns with the approach that\textsl{ not every student will be satisfied.} Our report is written under the assumption that the same choices for lunch will be presented to every student. It is unfortunately inevitable that there will be fringe cases where a student will not receive a satisfactory lunch. Thus, in our report,  the first goal is changed from the original question. 

Since the second concern was to determine the distribution of U.S. high school students, we felt that our report could be made more applicable if it only dealt with US high school students. An important rule this report will be working under is that the aforementioned student is between ages 14 and 18. In the United States, compulsory education is determined by individual states \cite{aragon_2015}. Most states end compulsory education at age 18. Subtracting 4 years of high school education gives us a starting age of 14. All references to students should mean a US high school student between ages 14 and 18.

For the purposes of this paper, "developing lifelong healthy eating habits" is done by intro

\chapter{What to Eat}\label{What to Eat}

\textbf{What Determines What Foods You Need?}

There are many factors that go into determining an individuals food requirements. Popular ones include height, weight, lifestyle, and gender. Designing lunches for students, however, is more complicated because we need to fit a standard set of meals to as many students as we can. This report will examine two factors that are directly concerned with the student: Nutrition and Fill. In this section we assume that students are being properly fed outside of school. It is outside the scope of this report to determine what constitutes malnutrition or obesity for students and how school lunches can address those problems. We design our school lunch as one part of the balanced diet. 

\section{Nutrition}\label{Nutrition}
The USDA concerns seem mostly oriented towards meeting caloric needs. Studies, however, have shown that calorie counting alone is not an effective diet for healthy living, especially for weight loss \cite{heymsfield}. This report recognizes the need for some basic nutritional needs to be met. Table \ref{NutritionGoal} is an excerpt from Table A7-1 from \cite{us20152015} with the three major macronutrients included. The 2015-2020 Dietary Guidelines for Americans are used by policymakers to design and implement programs such as USDA's National School Lunch Program. It is grounded in the most current scientific evidence and is developed by HHS and USDA nutrition and health experts to focus on disease prevention.  

\subsection{Calories}\label{Calories}
Calorie goals are perhaps the easiest thing to calculate. Table \ref{CalorieReq} is an excerpt of Table A2-1 from \cite{USDAHHKA} in our concerned age groups. The median and mode of these caloric requirements across all genders, activity levels, and ages is $2400$ while the mean is $2373.33$.

\begin{table}
\centering
\caption{Male/Female Calorie Requirements}
\label{CalorieReq}
\begin{tabular}{@{}|l|llll|l|lll@{}}
\cmidrule(r){1-1} \cmidrule(lr){6-6}
\textbf{MALES} &                                         &                                        &                                      &  & \textbf{FEMALES} &                                         &                                        &                                      \\ \cmidrule(r){1-4} \cmidrule(l){6-9} 
\textbf{AGE}   & \multicolumn{1}{l|}{\textbf{Sedentary}} & \multicolumn{1}{l|}{\textbf{Moderate}} & \multicolumn{1}{l|}{\textbf{Active}} &  & \textbf{AGE}     & \multicolumn{1}{l|}{\textbf{Sedentary}} & \multicolumn{1}{l|}{\textbf{Moderate}} & \multicolumn{1}{l|}{\textbf{Active}} \\ \cmidrule(r){1-4} \cmidrule(l){6-9} 
\textbf{14}    & \multicolumn{1}{l|}{2,000}              & \multicolumn{1}{l|}{2,400}             & \multicolumn{1}{l|}{2,800}           &  & \textbf{14}      & \multicolumn{1}{l|}{1,800}              & \multicolumn{1}{l|}{2,000}             & \multicolumn{1}{l|}{2,400}           \\ \cmidrule(r){1-4} \cmidrule(l){6-9} 
\textbf{15}    & \multicolumn{1}{l|}{2,200}              & \multicolumn{1}{l|}{2,600}             & \multicolumn{1}{l|}{3,000}           &  & \textbf{15}      & \multicolumn{1}{l|}{1,800}              & \multicolumn{1}{l|}{2,000}             & \multicolumn{1}{l|}{2,400}           \\ \cmidrule(r){1-4} \cmidrule(l){6-9} 
\textbf{16}    & \multicolumn{1}{l|}{2,400}              & \multicolumn{1}{l|}{2,800}             & \multicolumn{1}{l|}{3,200}           &  & \textbf{16}      & \multicolumn{1}{l|}{1,800}              & \multicolumn{1}{l|}{2,000}             & \multicolumn{1}{l|}{2,400}           \\ \cmidrule(r){1-4} \cmidrule(l){6-9} 
\textbf{17}    & \multicolumn{1}{l|}{2,400}              & \multicolumn{1}{l|}{2,800}             & \multicolumn{1}{l|}{3,200}           &  & \textbf{17}      & \multicolumn{1}{l|}{1,800}              & \multicolumn{1}{l|}{2,000}             & \multicolumn{1}{l|}{2,400}           \\ \cmidrule(r){1-4} \cmidrule(l){6-9} 
\textbf{18}    & \multicolumn{1}{l|}{2,400}              & \multicolumn{1}{l|}{2,800}             & \multicolumn{1}{l|}{3,200}           &  & \textbf{18}      & \multicolumn{1}{l|}{1,800}              & \multicolumn{1}{l|}{2,000}             & \multicolumn{1}{l|}{2,400}           \\ \cmidrule(r){1-4} \cmidrule(l){6-9} 
\end{tabular}
\end{table}

\subsection{Macronutrients}\label{Macronutrients}
There are three main types of nutrients called macro-nutrients that are essential for normal growth and survival: protein, carbohydrates, and fat.  Luckily, the Dietary Guidelines say that for our target students, the recommended percentages are the same. 



Many studies suggest that males and females in our target age range living an average lifestyle require a ratio of around 45:30:25 of Carbohydrates, Fats, and Proteins. This means that 45\% of calories will come from Carbohydrates, 30\% of calories from Fats, and 25\% of calories from Proteins. Applying this ratio for male female calorie requirements we get: 

The closest that the SELFNutritionData website can get to this ratio of 45:30:25 is 46:28:26. The results can be found at 

\texttt{http://nutritiondata.self.com/foods-000999000000046028026.html}

This means that in order to satisfy our macronutrient requirements, any food being served must be on that list. This approach simplifies the search for a mixture of foods that make up the perfect blend of macro nutrients because we will only be selecting from foods that \textit{already contain} the correct blend of macro nutrients. This approach is also used by the military in designing the Meal, Ready to Eat (MRE). Warfighters who do not intend to eat the entire MRE are instructed to eat a little bit of everything because the foods were selected with proper nutrient blends. The list of foods is exhaustive enough that there should be no problem finding variety or cost effective items. The list also includes many name brand items that are made with the same ingredients that school lunch caterers use. 

\begin{table}[]
\centering
\caption{
Nutritional Goals for Age-Sex Groups Based on Dietary Reference Intakes and Dietary Guidelines Recommendations
}
\label{NutritionGoal}
\begin{tabular}{@{}llll@{}}
\toprule
                                                & Source of goal\footnote{TEST TEST}            & Female 14-18                       & Male 14-18                            \\ \midrule
\multicolumn{1}{|l|}{Calorie level(s) assessed} & \multicolumn{1}{l|}{}     & \multicolumn{1}{l|}{1800}          & \multicolumn{1}{l|}{2200, 2800, 3200} \\ \midrule
\multicolumn{1}{|l|}{Macronutrients}            & \multicolumn{1}{l|}{}     & \multicolumn{1}{l|}{}              & \multicolumn{1}{l|}{}                 \\ \midrule
\multicolumn{1}{|l|}{Protein, g}                & \multicolumn{1}{l|}{RDA}  & \multicolumn{1}{l|}{46}            & \multicolumn{1}{l|}{52}               \\ \midrule
\multicolumn{1}{|l|}{Protein, \% kcal}          & \multicolumn{1}{l|}{AMDR} & \multicolumn{1}{l|}{10-30}         & \multicolumn{1}{l|}{10-30}            \\ \midrule
\multicolumn{1}{|l|}{Carbohydrate, g}           & \multicolumn{1}{l|}{RDA}  & \multicolumn{1}{l|}{130}           & \multicolumn{1}{l|}{130}              \\ \midrule
\multicolumn{1}{|l|}{Carbohydrate, \% kcal}     & \multicolumn{1}{l|}{AMDR} & \multicolumn{1}{l|}{45-65}         & \multicolumn{1}{l|}{45-65}            \\ \midrule
\multicolumn{1}{|l|}{Added sugars, \% kcal}     & \multicolumn{1}{l|}{DGA}  & \multicolumn{1}{l|}{<10\%} & \multicolumn{1}{l|}{<10\%}    \\ \midrule
\multicolumn{1}{|l|}{Total fat, \% kcal}        & \multicolumn{1}{l|}{AMDR} & \multicolumn{1}{l|}{25-35}         & \multicolumn{1}{l|}{25-35}            \\ \midrule
\multicolumn{1}{|l|}{Saturated fat, \% kcal}    & \multicolumn{1}{l|}{DGA}  & \multicolumn{1}{l|}{<10\%}         & \multicolumn{1}{l|}{<10\%}                        \\ \bottomrule
\end{tabular}
\end{table}

\subsubsection{Protein}\label{Protein}
Proteins are made up of amino acids. The human body cannot synthesize certain amino acids so it must obtain them through food \cite{Macro}. 

Most foods contain a mixture of different percentages of these essential amino acids. To assess the protein quality of each food we use SELFNutritionData's Protein Quality Indicator\textsuperscript{TM} (PQI). The PQI reports the percentage of the optimal level for the essential amino acids\footnote{Note that this is only the percentage breakdown for that food. You need to scale up the amounts of amino acids the specific protein contains to achieve the protein requirements. So if a }. The amino acid with the lowest level is considered the "`limiting"' amino acid for that food and determines the overall Amino Acid Score. To compensate for an incomplete Amino Acid Score, other proteins high in the limiting amino acid are required. 

\section{Fill}\label{Fill}

To determine how much a food will fill up the student, this report uses the Nutritional Target Map from SELFNutritionData. 

\chapter{A Sample Menu}\label{A Sample Menu}

\textbf{Menu Creation}

The specific food requirements listed in this report were intentionally made vague. Menu creation is a micro level requirement that should be determined by those who know their budgets and students the best: individual schools and their districts. It is impossible for this report to determine for schools exactly which foods they should purchase and serve. Any food that meets the stated requirements and individualized budgets can be used in a school lunch. Criticism of the Healthy, Hunger-Free Kids Act of 2010 included claims that the new requirements are too stringent, turning away students and costing schools more \cite{craiggundersen2014,juliekellyjeffstier2015,mythvsfact}. This report attempts to alleviate some of those concerns by giving some decisions back to the schools while still ensuring students are adequately fed. 

\chapter{Future Work}\label{Future Work}
One area that this report fails to address is customized meals. Even if individualized meals can not be served, meals can still be customized within region, state, county, school, and even down to grade if resources permit. An easy means of tailoring this model could involve adjusting the distributions and numbers for the target group of students. School lunch planners could examine specific BMI and nutritional needs to customize the standard lunch for that group. 


\addtocontents{toc}{\protect\vspace*{\baselineskip}}

\addcontentsline{toc}{chapter}{Bibliography} 
\nocite{*}
\bibliographystyle{plain}  
\bibliography{references}

\end{document}

